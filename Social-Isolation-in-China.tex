% Options for packages loaded elsewhere
\PassOptionsToPackage{unicode}{hyperref}
\PassOptionsToPackage{hyphens}{url}
\PassOptionsToPackage{dvipsnames,svgnames,x11names}{xcolor}
%
\documentclass[
  letterpaper,
  DIV=11,
  numbers=noendperiod]{scrartcl}

\usepackage{amsmath,amssymb}
\usepackage{lmodern}
\usepackage{iftex}
\ifPDFTeX
  \usepackage[T1]{fontenc}
  \usepackage[utf8]{inputenc}
  \usepackage{textcomp} % provide euro and other symbols
\else % if luatex or xetex
  \usepackage{unicode-math}
  \defaultfontfeatures{Scale=MatchLowercase}
  \defaultfontfeatures[\rmfamily]{Ligatures=TeX,Scale=1}
\fi
% Use upquote if available, for straight quotes in verbatim environments
\IfFileExists{upquote.sty}{\usepackage{upquote}}{}
\IfFileExists{microtype.sty}{% use microtype if available
  \usepackage[]{microtype}
  \UseMicrotypeSet[protrusion]{basicmath} % disable protrusion for tt fonts
}{}
\makeatletter
\@ifundefined{KOMAClassName}{% if non-KOMA class
  \IfFileExists{parskip.sty}{%
    \usepackage{parskip}
  }{% else
    \setlength{\parindent}{0pt}
    \setlength{\parskip}{6pt plus 2pt minus 1pt}}
}{% if KOMA class
  \KOMAoptions{parskip=half}}
\makeatother
\usepackage{xcolor}
\setlength{\emergencystretch}{3em} % prevent overfull lines
\setcounter{secnumdepth}{-\maxdimen} % remove section numbering
% Make \paragraph and \subparagraph free-standing
\ifx\paragraph\undefined\else
  \let\oldparagraph\paragraph
  \renewcommand{\paragraph}[1]{\oldparagraph{#1}\mbox{}}
\fi
\ifx\subparagraph\undefined\else
  \let\oldsubparagraph\subparagraph
  \renewcommand{\subparagraph}[1]{\oldsubparagraph{#1}\mbox{}}
\fi


\providecommand{\tightlist}{%
  \setlength{\itemsep}{0pt}\setlength{\parskip}{0pt}}\usepackage{longtable,booktabs,array}
\usepackage{calc} % for calculating minipage widths
% Correct order of tables after \paragraph or \subparagraph
\usepackage{etoolbox}
\makeatletter
\patchcmd\longtable{\par}{\if@noskipsec\mbox{}\fi\par}{}{}
\makeatother
% Allow footnotes in longtable head/foot
\IfFileExists{footnotehyper.sty}{\usepackage{footnotehyper}}{\usepackage{footnote}}
\makesavenoteenv{longtable}
\usepackage{graphicx}
\makeatletter
\def\maxwidth{\ifdim\Gin@nat@width>\linewidth\linewidth\else\Gin@nat@width\fi}
\def\maxheight{\ifdim\Gin@nat@height>\textheight\textheight\else\Gin@nat@height\fi}
\makeatother
% Scale images if necessary, so that they will not overflow the page
% margins by default, and it is still possible to overwrite the defaults
% using explicit options in \includegraphics[width, height, ...]{}
\setkeys{Gin}{width=\maxwidth,height=\maxheight,keepaspectratio}
% Set default figure placement to htbp
\makeatletter
\def\fps@figure{htbp}
\makeatother

\usepackage{booktabs}
\usepackage{longtable}
\usepackage{array}
\usepackage{multirow}
\usepackage{wrapfig}
\usepackage{float}
\usepackage{colortbl}
\usepackage{pdflscape}
\usepackage{tabu}
\usepackage{threeparttable}
\usepackage{threeparttablex}
\usepackage[normalem]{ulem}
\usepackage{makecell}
\usepackage{xcolor}
\KOMAoption{captions}{tableheading}
\makeatletter
\makeatother
\makeatletter
\makeatother
\makeatletter
\@ifpackageloaded{caption}{}{\usepackage{caption}}
\AtBeginDocument{%
\ifdefined\contentsname
  \renewcommand*\contentsname{Table of contents}
\else
  \newcommand\contentsname{Table of contents}
\fi
\ifdefined\listfigurename
  \renewcommand*\listfigurename{List of Figures}
\else
  \newcommand\listfigurename{List of Figures}
\fi
\ifdefined\listtablename
  \renewcommand*\listtablename{List of Tables}
\else
  \newcommand\listtablename{List of Tables}
\fi
\ifdefined\figurename
  \renewcommand*\figurename{Figure}
\else
  \newcommand\figurename{Figure}
\fi
\ifdefined\tablename
  \renewcommand*\tablename{Table}
\else
  \newcommand\tablename{Table}
\fi
}
\@ifpackageloaded{float}{}{\usepackage{float}}
\floatstyle{ruled}
\@ifundefined{c@chapter}{\newfloat{codelisting}{h}{lop}}{\newfloat{codelisting}{h}{lop}[chapter]}
\floatname{codelisting}{Listing}
\newcommand*\listoflistings{\listof{codelisting}{List of Listings}}
\makeatother
\makeatletter
\@ifpackageloaded{caption}{}{\usepackage{caption}}
\@ifpackageloaded{subcaption}{}{\usepackage{subcaption}}
\makeatother
\makeatletter
\@ifpackageloaded{tcolorbox}{}{\usepackage[many]{tcolorbox}}
\makeatother
\makeatletter
\@ifundefined{shadecolor}{\definecolor{shadecolor}{rgb}{.97, .97, .97}}
\makeatother
\makeatletter
\makeatother
\ifLuaTeX
  \usepackage{selnolig}  % disable illegal ligatures
\fi
\IfFileExists{bookmark.sty}{\usepackage{bookmark}}{\usepackage{hyperref}}
\IfFileExists{xurl.sty}{\usepackage{xurl}}{} % add URL line breaks if available
\urlstyle{same} % disable monospaced font for URLs
\hypersetup{
  pdftitle={The Spatial Relationship between Social Media, Digital Connectedness, and Social Isolation in China: Revisting Cyber-Optimism},
  pdfauthor={Jason Gainous; Andrew W. MacDonald},
  pdfkeywords={social isolation, social media, digital
connectedness, China},
  colorlinks=true,
  linkcolor={blue},
  filecolor={Maroon},
  citecolor={Blue},
  urlcolor={Blue},
  pdfcreator={LaTeX via pandoc}}

\title{The Spatial Relationship between Social Media, Digital
Connectedness, and Social Isolation in China: Revisting Cyber-Optimism}
\author{Jason Gainous\footnote{University of Louisville,
  jason.gainous@louisville.edu} \and Andrew W. MacDonald\footnote{Duke
  Kunshan University, andrew.macdonald@dukekunshan.edu.cn}}
\date{3/5/23}

\begin{document}
\maketitle
\begin{abstract}
Early research and populuar discourse on the promise of the internet was
optimistic. Many believed it would promote social interconnectedness
and, ultimately, be a positive force in people's lives. Since these
early predictions, research has become much less positive, usually
pointing to the negative aspects of social media. Here we argue that the
digital environment can actually create a space that deters feelings of
social isolationism especially for those living outside cities in
restrictive information environments where citizens are often
disconnected from the outside world. Using data from a nationwide survey
of Chinese citizens we conducted in 2015, we find that the more citizens
digitally connect with others, the more likely they are to move
relationships from online to offliine, and the more they use social
media the less personally isolated they feel. Futher, those living in
urban environments tend to feel more isolated, and this spatial factor
seems to moderate the aforementioned relationships. All three of those
observed relationships are stronger for those in less urban
environments.
\end{abstract}
\ifdefined\Shaded\renewenvironment{Shaded}{\begin{tcolorbox}[breakable, enhanced, frame hidden, interior hidden, boxrule=0pt, sharp corners, borderline west={3pt}{0pt}{shadecolor}]}{\end{tcolorbox}}\fi

\hypertarget{social-isolation-in-china}{%
\section{Social Isolation in China}\label{social-isolation-in-china}}

\hypertarget{introduction}{%
\subsection{Introduction}\label{introduction}}

Research is quite mixed when it comes to identifying the relationship
between social media use and feelings of isolationism. Early research
suggested that the internet would build social capital, or an
interconnectedness between people, and that social media platforms were
the ultimate way to bring people together. Since then, research
indicates that the effect may be the opposite for many. Social media may
encourage social isolation, and ultimately, social media users may feel
more disconnected from others than those who do not use social media. We
know very little about this phenomenon in China. China presents an
interesting case because the internet is bifurcated to some degree --
there is the Chinese internet and the outside internet available via
VPNs. Being isolation to the outside world may lessen feelings of
isolation, or it may increase these feelings because digital exposure to
a world is much larger than that in people's immediate environment
without the ability to have in-person connection to may make one feel
isolated. Interestingly, and somewhat ironically though, it may also
matter, too, whether someone lives in a city. Research has suggested
that urban living can actually lead to feelings of isolation.

\hypertarget{introduction-1}{%
\subsection{Introduction}\label{introduction-1}}

Research is quite mixed when it comes to identifying the relationship
between social media use and feelings of isolationism. Early research
suggested that the internet would build social capital, or an
interconnectedness between people, and that social media platforms were
the ultimate way to bring people together. Since then, research
indicates that the effect may be the opposite for many. Social media may
encourage social isolation, and ultimately, social media users may feel
more disconnected from others than those who do not use social media. We
know very little about this phenomenon in China. China presents an
interesting case because the internet is bifurcated to some degree --
there is the Chinese internet and the outside internet available via
VPNs. Being isolation to the outside world may lessen feelings of
isolation, or it may increase these feelings because digital exposure to
a world is much larger than that in people's immediate environment
without the ability to have in-person connection to may make one feel
isolated. Interestingly, and somewhat ironically though, it may also
matter, too, whether someone lives in a city. Research has suggested
that urban living can actually lead to feelings of isolation.

\hypertarget{data}{%
\subsection{Data}\label{data}}

Our data are based on a random Internet survey conducted by Qualtrics
from November 25 to December 2, 2015. We designed the instrument,
content and structure. After we designed the instrument in English and
translated it into Chinese, we entered the questions into the graphical
interface provided from our survey provider, Qualtrics. We were careful
at this stage to make the formatting clear, and to distribute the
questions across numerous pages to prevent overwhelming respondents with
too many questions at once. This helped us achieve a high completion and
response rate. Additionally, we were particularly attentive to the
possibility of question order effects. Many of the questions are
sensitive and could cue responses to subsequent questions. We grouped
and placed questions strategically to limit any such potential for
response bias. Once we settled on the design, Qualtrics administered the
survey in November 2015. They recruit a large pool of respondents for
various survey projects through online advertising on websites such as
local portals, search engines, social networking services, and/or online
shopping services. Panelists who update their profiles at least once
every 6 months are randomly invited to participate in surveys for which
they qualify. Online points are awarded to the corresponding panelists,
which they may accumulate and exchange for cash or various other
country-specific gifts to serve as an incentive for completion. This
number of points is based on the length of the survey. Ours had over
fifty questions, so respondents received a relatively high number of
points.

The data collection had two phases. For the first phase, Qualtrics
collected and submitted to us a trial run of 286 cases. We were able to
check the data for reliability and adjust the instrument before
proceeding with the second phase, the final data collection. We only
made three adjustments, all of which turned out to be important and
central to our analysis. For the trial run, we only asked respondents
about pro-government Internet posters and did not ask them about
potential net-spy or hostile posters. We amended this oversight and it
became central to the analysis that follows. In addition, much of our
theory focuses on the impact of Chinese Internet users circumventing
government filters on the Internet. We initially had a question about
whether citizens had jumped the wall to read sensitive political
information (see Appendix A -- Q25), but decided to add whether they had
done so for entertainment purposes as well (Q26 - to watch foreign
movies, television shows, etc.). Finally, we also included an attention
filter question as part of a battery of institutional trust questions
where respondents were simply told to select the ``None at all''
response (Q13H). If they did not select this response, we could assume
they were not paying attention to the questions, and the survey ended
for these respondents and the data were not collected for them. In the
end, the full sample included a total of 2292 respondents. This sample
size provides for a roughly ±2 margin of error.

While our sample was randomly selected, the sampling frame was based
only on Internet users that Qualtrics can access. Qualtrics recruits
their frame subjects through online advertising on websites such as
local portals, search engines, social networking services, and/or online
shopping services. We do not contend that our sample is representative
of the Chinese population, as the method of obtaining respondents
obviously greatly limits those citizens with little or no access to the
Internet. That said, we are comfortable that our sample is generally
representative of Chinese Internet users. Given that our research focus
centers on Internet effects, it makes sense to have a large sample of
Internet users. A sampling frame including the entire Chinese population
did not make sense for this research. Half of the population do not
regularly access the Internet according to World Bank and China Internet
Network Information Center (CNNIC) data as of late 2015 when we
conducted our survey. Using a sample of the entire population could
potentially limit the degrees of freedom in our models and the marginals
of nuanced measures of digital media use would likely become too small
to have any value. Thus, the large sample of Internet users adds to our
confidence in the inferences throughout the book.



\end{document}
